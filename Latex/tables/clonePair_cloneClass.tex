% !TeX root = ../main.tex
% Add the above to each chapter to make compiling the PDF easier in some editors.
\setlength{\extrarowheight}{.5em}
\begin{table}
	\annotation{I dont know how to make code snippets out of it. Tried many ways but listings in tables is somewhat awful.}
	\caption{Clone Pair and Clone Class}
	\label{table:cloneClass}

%	\begin{tabular}{l|l}
%		\begin{lstlisting}[frame=none, numbers=none]
%			public void test(){ ...	}
%		\end{lstlisting} 
%		
%		&
%		
%		\begin{lstlisting}[frame=none, numbers=none]
%		public void test(){ ...	}
%		\end{lstlisting}
%	\end{tabular}		
	\begin{tabularx}{1.0\textwidth}{C|C|C}
%		\hline
		\textbf{Fragment 1} & \textbf{Fragment 2} & \textbf{Fragment 3} \\
		\hline
		
		\cellcolor{TUMAccentLightBlue}/* The following code was generated by \dots */ &
		\cellcolor{TUMAccentLightBlue}/* The following code was generated by \dots */ &
		/** This character denotes the end of file */ \\
		\hline
				
		\cellcolor{TUMAccentBlue}/* The content of this method is always regenerated */ &
		\cellcolor{TUMAccentBlue}/* The content of this method is always regenerated */ &
		\cellcolor{TUMAccentBlue}/* The content of this method is always regenerated */ \\
		\hline
		
		/** Translates characters to character classes */ &
		\cellcolor{TUMAccentLightBlue} /** This class is a scanner generated by \dots */ &
		\cellcolor{TUMAccentLightBlue} /** This class is a scanner generated by \dots */ \\
	\end{tabularx}
\end{table}