% !TeX root = ../main.tex
% Add the above to each chapter to make compiling the PDF easier in some editors.
\setlength{\extrarowheight}{.5em}
\begin{table}
	\caption{Token properties}
	\label{table:tokenProperties}
	\begin{tabularx}{\textwidth}{c|L}
		\textbf{Property} & \textbf{Description} \\
		\hline
		Text & The original input text of the token, copied verbatim from the source. \\
		Offset & The number of characters before this token in the text. The offset is 0-based and inclusive. \\
		End Offset & The number of characters before the end of this token in the text (i.e. the 0-based index of the last character, inclusive). \\
		Origin ID & The string that identifies the origin of this token. This can, e.g., be a uniform path to the resource. Its actual content depends on how the token gets constructed. \\
		Type & The type of the token. \\
		Language & The programming language in which the source code file is written that contains the token.		
	\end{tabularx}
\end{table}
\setlength{\extrarowheight}{0em}