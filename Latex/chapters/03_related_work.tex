% !TeX root = ../main.tex
% Add the above to each chapter to make compiling the PDF easier in some editors.
\chapter{Related Work}\label{chapter:related}
Our work is done in a field where much research is performed. This makes our work easily comparable to other ones, especially the one provided in the paper presented in the following subsections.
\section{Automatic Categorization of Source Code in Open-Source Software - Jonathan Bernwieser}
In his thesis, Johnathan Bernwieser introduced an attempt of automating the process of categorization of source code in \cite{Bernwieser2014}. He classified source code in \textit{productive} and \textit{test} code, followed by a sub-categorization in \textit{manually maintained} and \textit{automatically generated}.\\
He ran in several problems during his research, especially with the identification of generated code. Different to test code, which often follows the naming convention to include the word \textit{test} into the class name or that the test classes do follow inheritance lines which identify them, generated code has no standardized way of being marked by the respective code generator.\\
In his approach he used a filtering of the classes which used the observation that code generators often include the term \textit{generated by} in their comments. It quickly turned out that this approach generated many false positives as software developers often also use this term in their documentation. Nevertheless he pointed out that further investigation on the comments, especially in the ones including this term, might result in another way to find generated code. This thesis deploys his finding. \\
Due to the high number of false-positive categorizations of generated code resulting from this simple assumption, Bernwieser tackled the problem by using clone-detection on source code. He examined the question if a high clone coverage respectively higher clone density implies generated classes.\\
Therefore he implemented two mechanisms that examined source code for the following clone metrics.

\subsection{Number of clones per clone class}
This metric gave very interesting results on the data set used by Bernwieser. It turned out that generated code often has many instances among projects due to the usage of templates and routines from which it gets generated and what makes the resulting source code identically between generations. The problem that he ran into was the fact that there exists also generated code with only a small number of instances which resulted in a minimum filter threshold that had to be really low to detect all generated classes, which made it impractical to use.

\subsection{Clone coverage per class}
In this approach Bernwieser tried different clone coverage thresholds and lines of code limits to detect generated code. But the resulting problem is that there exists no exact correlation between the size of a source code file and the possibility of it containing generated code. Furthermore it showed that the size of a source code file is irrelevant for being a generated class. This approach produced a high number of false negatives and thus filtered out generated code which should have been detected.

\subsection{Results}
Bernwieser showed that there is indeed a correlation between clone coverage and code generation, but at this point there was no way found on how to use this correlation to automate code generator identification. Anyhow he pointed out the fact that many code generators add specific comments to the generated files he found and on which he depicted further research could be done.

\subsection{Differences in the approaches}
In this thesis, the approach differs from Bernwiesers so that the clone-detection is done on the comments added to the source code files in the data set, rather than on the respective source code. A goal of our approach is to find the specific generator patterns that Bernwieser pointed at. It's based on his observation that the same code generators do always add the same characteristic generator patterns regardless of what template or routine the generator used for generation. Resulting on this the target of this thesis is to find these patterns by using clone-detection on comments. The expected result is that the clone classes with the highest number of instances will be the specific generator patterns.

