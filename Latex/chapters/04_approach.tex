% !TeX root = ../main.tex
% Add the above to each chapter to make compiling the PDF easier in some editors.

\chapter{Approach}\label{chapter:approach}
One aim of this thesis is the prototypical implementation of heuristics to detect generated code which uses techniques of clone-detection on the comments that are extracted from the source code. A list of patterns that detect generated code of several code generators is derived based on comments in source code and we created a generator-pattern repository. This repository provides a database of code generators and their respective characteristic generator pattern which identifies the the generated source code.\\
We use a teamscale server to perform the basic analysis tasks on the specified projects.
\section{Use Teamscale as Lexer to extract Tokens}
The first step in the approach used in this paper is the lexical analysis of the source code included in the projects. Teamscale comes with a multitude of lexers applicable for many different programming languages. The general workflow in this step is reading each source code file found in the project, perform the lexical analysis and tokenize the source code file into a sequence of logically coherent tokens representing the source code on a logical level. The tokens are saved server-side in the Teamscale instance.
% !TeX root = ../main.tex
% Add the above to each chapter to make compiling the PDF easier in some editors.

\begin{center}
	\begin{tikzpicture}[scale=2, node distance = 4cm, auto]
		\node [block] (init) {Source Code File};
		\node [block, right of=init] (lexer) {Lexical Analysis};
		\node [block, right of=lexer] (tokens) {Sequence of Tokens};
		% Draw edges
		\path [line] (init) -- (lexer);
		\path [line] (lexer) -- (tokens);
	\end{tikzpicture}
\end{center}

\section{Connect to the Teamscale server and retrieve comments}
\label{section:retrieveComments}
In the second step the tool written for this thesis connects to the Teamscale instance and retrieves the comments.\\
This has to be done in the following steps:
\begin{enumerate}
	\item Get the projects that are currently available in the Teamscale instance. 
	\item For each project retrieve the respective uniform paths\footnote{A uniform path is the path the Teamscale instance uses to represent the source code file. The URL to access the file on the server can be build using it.} to access the source code files on the server.
	\item For each uniform path retrieve the comments that are included in the respective source code file. In this step the server filters all tokens that are available for each file. The only important token class is the \textit{COMMENT} class, which itself contains the sub-classes shown in Table \ref{table:commentTypes}.
	\item The local file path of every source code file is retrieved from the server based on the uniform path. This will get important later in the evaluation.
	\item Each local file path gets associated to the respective \code{List} of comments.
\end{enumerate}
% !TeX root = ../main.tex
% Add the above to each chapter to make compiling the PDF easier in some editors.
\begin{center}
	\begin{tikzpicture}[scale=2, node distance = 3cm, auto]
		\node [block] (init) {Get Projects};
		\node [block, right of=init] (uniforms) {Retrieve Uniform Paths};
		\node [block, right of=uniforms] (filter) {Filter and Transfer Comments};
		\node [block, right of=filter] (local) {Get Local File Paths};
		\node [block, right of=local] (associate) {Associate Path to Comments};
		
		\path [line] (init) -- (uniforms);
		\path [line] (uniforms) -- (filter);
		\path [line] (filter) -- (local);
		\path [line] (local) -- (associate);
	\end{tikzpicture}
\end{center}
\threaded{Retrieving the comments}
% !TeX root = ../main.tex
% Add the above to each chapter to make compiling the PDF easier in some editors.

\begin{table}[H]
	\caption{Comment types}
	\label{table:commentTypes}
	\begin{tabularx}{\textwidth}{l|L}
		\textbf{Token type} & \textbf{Example} \\
		\hline
		HASH\_COMMENT & \textit{\# Sample PHP script accessing HyperSQL through the ODBC extension module.}\\ 
		DOCUMENTATION\_COMMENT & \textit{/** Generated By:JJTree: Do not edit this line. */} \\ 
		SHEBANG\_LINE & \textit{\#!/usr/bin/env python } \\ 
		TRIPLE\_SLASH\_DIRECTIVE & \textit{/// <reference path="Parser.ts" />} \\ 
		TRADITIONAL\_COMMENT & \textit{/* Do not modify this code */} \\ 
		MULTILINE\_COMMENT & \textit{"""Testsuite for TokenRewriteStream class."""} \\ 
		END\_OF\_LINE\_COMMENT & \textit{// don't care about docstrings} \\ 
		SIX\_COLUMNS\_COMMENT & \textit{\dots} \\ 
	\end{tabularx} 
\end{table}

\section{Filter generated files}
With every iteration of the algorithm the generator-pattern repository grows and more patterns are found. These patterns are used to filter the already as generated detected source code files to reduce the amount of work that has to be done by the following steps. \\
Additionally, the amount of found presumable generator-patterns is reduced drastically when we sort out the generated files before applying the follwing steps.

\section{Prepare comments for suffix-tree clone-detection}
Once the comments are received from the Teamscale server instance they have to get prepared to be usable in the suffix-tree clone-detection as introduced by Esko~Ukkonen~\cite{Ukkonen1995}\cite{Ukkonen1993}.\\
\threaded{The preparation}

\subsection{Create thread for each source code file}
The step described in Section~\ref{section:retrieveComments} produces a \code{Map} in which every local path to a source code file is associated with its respective \code{List} of comments.
In the first part of the preparation one thread is generated for each source code file respective local file path. These threads will be responsible for the preparation of each single file and the results will be merged at the end.

\subsection{Convert comments into CloneChunks}
At first the comments are split at the linebreaks into single lines and leading and trailing whitespaces of each line are removed. The resulting lines are in turn split at the whitespaces separating single words and expressions.\\
The \code{List} of words gets now iterated and every word gets checked if it holds any valuable information. To evaluate the value of each word it is checked if it contains alphabetic letters \textit{(a-z, A-Z)} or digits. If a word only consists of nonalphabetic characters and no digits it gets sorted out \textit{(e.g. "-{}-{}-{}-{}-{}-{}-{}-", "/**", "<!--")}.\\
The remaining words that bring a value to the approach are now converted into \code{CloneChunk}s. Therefore is added to every word the type of the comment, the offset and the line number of the comment and the local file path the comment originated from.\\
For an overview over these properties of a comment see \ref{section:token}.

\subsection{Add sentinels}
To prevent the suffix-tree clone-detection algorithm from intermixing the comments a sentinel~[\ref{section:sentinel}] gets added at the end of the \code{List} of \code{CloneChunk}s resulting from the previous steps. This is necessary due to the fact that the algorithm works an a single \code{List} of \code{CloneChunk}s that is build from merging all \code{CloneChunk}s from every comment.

\subsection{Merge thread results}
Once all threads are finished, resulting in all comments being converted into \code{List}s of \code{CloneChunk}s and sentinels being added to the end of each \code{List}, all \code{List}s get concatenated into one huge \code{List} containing all words of all comments from all projects. On this \code{List} the suffix-tree clone-detection algorithm can now be applied.

\section{Build suffix-tree}
Esko Ukkonen introduced an algorithm to construct suffix-trees on strings in his paper for \textit{Algorithmica} in 1995 \cite{Ukkonen1995}. "A suffix-tree is a trie-like structure representing all suffixes of a string. [\dots] The new algorithm [\dots] processes the string symbol by symbol from left to right [\dots]. The algorithm is based on the simple observation that the suffixes of a string $T^i = t_1 \dots t_i$ can be obtained from the suffixes of string $T^{i-1} = t_1 \dots t_{i-1}$ by catenating symbol $t_i$ at the end of each suffix of $T^{i-1}$ and by adding the empty suffix."~\cite{Ukkonen1995}\\
For this thesis the algorithm has been expanded to work on \code{List}s of arbitrary data types. The simple observation that lead to the expansion of the algorithm is that a string is just a \code{List} of characters. When building the suffix-tree this \code{List} gets traversed from left to right and each \code{CloneChunk} is added and the suffix-tree is expanded.\\
We used this property of the algorithm to extend it to a generic version of it by allowing \code{List}s of any data types. The only mandatory property the data type must provide is an equality function that can be evaluated in linear time to keep the linear time property of the algorithm. We implemented a function that calculates a hash value of the data structure and performs a simple \textit{integer - integer} comparison that keeps the linear time property.\\
Figure~\ref{fig:suffixTreeConstructionImplicit} illustrates the implicit suffix-tree on the sequence of \code{CloneChunk}s representing the \code{List} of words \textit{Do not modify \dots Do not modify}. After adding a sentinel at the end of the \code{List} the suffix-tree becomes explicit, which is used in this thesis and shown in Figure~\ref{fig:suffixTreeConstructionExplicit}.\\
For a deeper insight into the details of the algorithm for suffix-tree construction see \cite{Ukkonen1995}. 
% !TeX root = ../../main.tex
% Add the above to each chapter to make compiling the PDF easier in some editors.
\begin{center}
	
%1
%\begin{tikzpicture}[
%	scale = 1, transform shape, thick, TUMLightGray,
%	every node/.style = {draw, black, circle, minimum size = 1mm},
%	grow = down,  % alignment of characters
%	level 1/.style = {sibling distance=3cm},
%	level 2/.style = {sibling distance=2cm}, 
%	level 3/.style = {sibling distance=2cm}, 
%	level distance = 2cm
%	]
%	
%	\node[](Root){}
%		child{node (Do){}}
%	;
%		
%	% Labels
%	\begin{scope}[nodes = {TUMBlue, draw = none, sloped, anchor=center, above}]
%		\path (Root) -- (Do) node [near end, left] {Do};
%	\end{scope}
%\end{tikzpicture}
%
%% 2
%\begin{tikzpicture}[
%scale = 1, transform shape, thick, TUMLightGray,
%every node/.style = {draw, black, circle, minimum size = 1mm},
%grow = down,  % alignment of characters
%level 1/.style = {sibling distance=3cm},
%level 2/.style = {sibling distance=2cm}, 
%level 3/.style = {sibling distance=2cm}, 
%level distance = 2cm
%]
%
%\node[](Root){}
%child{node (Do not){}}
%child{node (not){}}
%;
%
%% Labels
%\begin{scope}[nodes = {TUMBlue, draw = none, sloped, anchor=center, above}]
%\path (Root) -- (Do not) node [near end, right] {Do not};
%\path (Root) -- (not) node [near end, left] {not};
%\end{scope}
%\end{tikzpicture}
%
%% 3
%\begin{tikzpicture}[
%scale = 1, transform shape, thick, TUMLightGray,
%every node/.style = {draw, black, circle, minimum size = 1mm},
%grow = down,  % alignment of characters
%level 1/.style = {sibling distance=3cm},
%level 2/.style = {sibling distance=2cm}, 
%level 3/.style = {sibling distance=2cm}, 
%level distance = 4cm
%]
%
%\node[](Root){}
%child{node (Do not modify){}}
%child{node (not modify){}}
%child{node (modify){}}
%;
%
%% Labels
%\begin{scope}[nodes = {TUMBlue, draw = none, sloped, anchor=center, above}]
%\path (Root) -- (Do not modify) node [near end, right] {Do not modify};
%\path (Root) -- (not modify) node [near end, left] {not modify};
%\path (Root) -- (modify) node [near end, left] {modify};
%\end{scope}
%\end{tikzpicture}
%
%% 4
%\begin{tikzpicture}[
%scale = 1, transform shape, thick, TUMLightGray,
%every node/.style = {draw, black, circle, minimum size = 1mm},
%grow = down,  % alignment of characters
%level 1/.style = {sibling distance=3cm},
%level 2/.style = {sibling distance=2cm}, 
%level 3/.style = {sibling distance=2cm}, 
%level distance = 4cm
%]
%
%\node[](Root){}
%child{node (Do not modify x){}}
%child{node (not modify x){}}
%child{node (modify x){}}
%child{node (x){}}
%;
%
%% Labels
%\begin{scope}[nodes = {TUMBlue, draw = none, sloped, anchor=center, below}]
%\path (Root) -- (Do not modify x) node [near start, left] {Do not modify ...};
%\path (Root) -- (not modify x) node [near start, left] {not modify ...};
%\path (Root) -- (modify x) node [near start, right] {modify ...};
%\path (Root) -- (x) node [near start, right] {...};
%\end{scope}
%\end{tikzpicture}
%
%% 5
%\begin{tikzpicture}[
%scale = 1, transform shape, thick, TUMLightGray,
%every node/.style = {draw, black, circle, minimum size = 1mm},
%grow = down,  % alignment of characters
%level 1/.style = {sibling distance=3cm},
%level 2/.style = {sibling distance=2cm}, 
%level 3/.style = {sibling distance=2cm}, 
%level distance = 5cm
%]
%
%\node[](Root){}
%child{node (Do not modify){}
%	child{node (x Do not modify S){}}
%	child{node (S){}}
%}
%child{node (not modify x){}}
%child{node (modify x){}}
%child{node (x){}}
%;
%
%% Labels
%\begin{scope}[nodes = {TUMBlue, draw = none, sloped, anchor=center, above}]
%\path (Root) -- (Do not modify) node [near start, left] {Do not modify};
%\path (Do not modify) -- (x Do not modify S) node [near start, left] {... Do not modify \$};
%\path (Do not modify) -- (S) node [near end, right] {\$};
%
%\path (Root) -- (not modify x) node [near start, left] {not modify ...};
%\path (Root) -- (modify x) node [near start, right] {modify ...};
%\path (Root) -- (x) node [near start, right] {...};
%\end{scope}
%\end{tikzpicture}

% 6
	\begin{figure}	
		\centering
		\begin{tikzpicture}[
		scale = 1, transform shape, thick, TUMLightGray,
		every node/.style = {draw, TUMBlue, circle, minimum size = 1mm},
		grow = down,  % alignment of characters
		level 1/.style = {sibling distance=3cm},
		level 2/.style = {sibling distance=2cm}, 
		level 3/.style = {sibling distance=2cm}, 
		level distance = 8cm
		]
		
		\node[](Root){}
		child{node (Do not modify){}}
		child{node (not modify){}}
		child{node (modify){}}
		child{node (E){}}
		;
		
		% Labels
		\begin{scope}[nodes = {Black, draw = none, sloped, anchor=center, below}]
		\path (Root) -- (Do not modify) node [near start, left] {Do not modify \dots Do not modify};
		
		\path (Root) -- (not modify) node [near start, left] {not modify \dots Do not modify};
		
		\path (Root) -- (modify) node [near start, right] {modify \dots Do not modify};
		
		\path (Root) -- (E) node [near start, right] { \dots Do not modify};
				
		\end{scope}
			
		\end{tikzpicture}
		\caption{The implicit suffix tree for the list of words "Do not modify \dots Do not modify". }
		\label{fig:suffixTreeConstructionImplicit}
	\end{figure}
\end{center}
% !TeX root = ../../main.tex
% Add the above to each chapter to make compiling the PDF easier in some editors.
\begin{center}	
	\begin{figure}	
		\centering
		\begin{tikzpicture}[
		scale = 1, transform shape, thick, TUMLightGray,
		every node/.style = {draw, TUMBlue, circle, minimum size = 1mm},
		grow = down,  % alignment of characters
		level 1/.style = {sibling distance=3cm},
		level 2/.style = {sibling distance=2cm}, 
		level 3/.style = {sibling distance=2cm}, 
		level distance = 5cm
		]
		
		\node[](Root){}
		child{node (S){}}
		child{node (Do not modify){}
			child{node (x Do not modify S 1){}}
			child{node (S 1){}}
		}
		child{node (not modify){}
			child{node (x Do not modify S 2){}}
			child{node (S 2){}}
		}
		child{node (modify){}
			child{node (x Do not modify S 3){}}
			child{node (S 3){}}
		}
		child{node (x Do not modify S){}}
		;
		
		% Labels
		\begin{scope}[nodes = {Black, draw = none, sloped, anchor=center, below}]
		\path (Root) -- (Do not modify) node [near start, left] {Do not modify};
		\path (Do not modify) -- (x Do not modify S 1) node [near start, left] {... Do not modify \$};
		\path (Do not modify) -- (S 1) node [near end, right] {\$};		
		\path (Root) -- (not modify) node [near start, right] {not modify};
		\path (not modify) -- (x Do not modify S 2) node [near start, left] {... Do not modify \$};
		\path (not modify) -- (S 2) node [near end, right] {\$};		
		\path (Root) -- (modify) node [near start, right] {modify};
		\path (modify) -- (x Do not modify S 3) node [near start, left] {... Do not modify \$};
		\path (modify) -- (S 3) node [near end, right] {\$};		
		\path (Root) -- (x Do not modify S) node [near start, right] {... Do not modify \$};
		\path (Root) -- (S) node [near end, right] {\$};				
		\end{scope}
			
		\end{tikzpicture}
		\caption{The explicit suffix tree for the list of words "Do not modify \dots Do not modify". }
		\label{fig:suffixTreeConstructionExplicit}
	\end{figure}
\end{center}

\section{Find clones}
\label{section:findClones}
To find duplicate sequences in the suffix-tree, a variety of tree search algorithms can be applied. They are all based on the observation that the clones are represented by the paths in the suffix-tree containing the searched sequence. Listing~\ref{lst:cloneBFS} can be used to perform a breadth first search to find clones.\\
When the graph search is performed the clone classes get returned as a \code{List} whereas each entry represents a \code{CloneClass}. Each \code{CloneClass} itself is a \code{List} of \code{List} of \code{CloneChunk}s. Each inner \code{List} of the \code{CloneClass} represents a clone instance. The clones consist of a \code{List} of \code{CloneChunk}s which have the additional information added to the string representation as shown in Listing~\ref{lst:cloneClass}. So every clone that is represented by the \code{List} named \textit{members} can be associated to the file it originated from. This will become important in the next steps.\\
This \code{List} of \code{CloneChunk}s gets now converted into a \code{List} of \code{CloneResult}. A \code{CloneResult} is a more compact and convenient representation of the clone classes. Each has the text it represents as its \textit{name} and a \code{List} of all occurrences associated, whereas a occurrence is a \code{Pair} representing the \texttt{originID} (file identifier) and the \texttt{line number}. \threaded{The conversion}
% !TeX root = ../main.tex
% Add the above to each chapter to make compiling the PDF easier in some editors.
\copied
\begin{lstlisting}[
	caption=Breadth first search algorithm to find all containing branches in a suffix tree for a given sequence.,
	label=lst:cloneBFS] 
	consider current vertex cur
	for (each child of current vertex cur)
		if (P does not match child.label)
			continue
	if (a full match) return all results
	if (a partial match) go deeper
	return no match	
\end{lstlisting}
% !TeX root = ../main.tex
% Add the above to each chapter to make compiling the PDF easier in some editors.

\begin{lstlisting}[
	caption=The clone classes as returned by the graph search algorithm.,
	label=lst:cloneClass]
{
  "cloneClasses": [
    "members": [
      ["Do", "not", "modify"],	
      ["Do", "not", "modify"],
      ["Do", "not", "modify"]],
    "members": [
      ["This", "class", "is", "generated", "by"],
      ["This", "class", "is", "generated", "by"],
      ["This", "class", "is", "generated", "by"]],
    "members": [
      ["The", "following", "code", "was", "generated", "by"],
      ["The", "following", "code", "was", "generated", "by"],
      ["The", "following", "code", "was", "generated", "by"]],
   ]
}
\end{lstlisting}{
% !TeX root = ../main.tex
% Add the above to each chapter to make compiling the PDF easier in some editors.

\begin{lstlisting}[
	caption=Clone results after processing of the clone classes.,
	label=lst:cloneResult]
{
  "name": "Do not modify", "occurrences": [
    [originID: "/projectX/class1.java", lineNumber: "9"]
    [originID: "/projectX/class2.java", lineNumber: "9"]
    [originID: "/projectX/class3.java", lineNumber: "9"]],
  "name": "This class is generated by", "occurrences": [
    [originID: "/projectY/class1.java", lineNumber: "1"]
    [originID: "/projectY/class2.java", lineNumber: "1"]
    [originID: "/projectY/class3.java", lineNumber: "1"]],
  "name": "The following code was generated by", "occurrences": [
    [originID: "/projectZ/class1.java", lineNumber: "48"]
    [originID: "/projectZ/class2.java", lineNumber: "48"]
    [originID: "/projectZ/class3.java", lineNumber: "48"]],
}
\end{lstlisting}

\section{Reduce amount of not valuable data}
\label{section:processCloneResults}
The raw suffix-tree clone-detection produces a lot of false positive \code{CloneResult}s. This is due to the fact that the clone-detection cannot distinguish between common phrases and actual generator-patterns. To reduce the amount of not valuable and redundant data three steps are performed.
\subsection{Filter possibly generated CloneResults}
\label{section:filterCloneResults}
In this steps the \code{CloneResult}s get filtered so that possibly generated ones can easily be distinguished from the ones that are unlikely to be generated. This step is only to improve and speed up the manual search for generator patterns in the found clones performed later.\\
As stated in \cite{Bernwieser2014} most generator patterns will include the word \textit{generated}. From this observation a filter pattern has been derived and extended which is shown in Listing~\ref{lst:generatorPattern}. With this pattern the \code{List} of \code{CloneResult} gets filtered by their names.
% !TeX root = ../main.tex
% Add the above to each chapter to make compiling the PDF easier in some editors.

\begin{lstlisting}[
	caption=Pattern to filter possibly generated clone classes.,
	label=lst:generatorPattern]
"(Do not (modify|edit|change))|(generate(d)?)"
\end{lstlisting}

\subsection{Accumulation of CloneResults}
\label{section:accumulation}
Due to the fact that the clone-detection on suffix-trees can only settle on a minimum clone length, but not on a maximum length, an accumulation is done.\\
As seen in Listing~\ref{lst:beforeAccumulation} the name of the first \code{CloneResult} is a substring of the names of the other two. We suspect that the sets of occurrences of the \code{CloneResult}s do overlap in this case, which is illustrated in Listing~\ref{lst:beforeAccumulation}. So to speed up and remove redundant data only the \code{CloneResult} with the shortest name length is kept and the sets of occurrences from all \code{CloneResult}s that include the name as a substring get merged into that \code{CloneResult}.\\
As illustrated in Listing~\ref{lst:afterAccumulation}, the amount of data stored is drastically reduced what speeds up processing later on.
% !TeX root = ../main.tex
% Add the above to each chapter to make compiling the PDF easier in some editors.

\begin{lstlisting}[
	caption=The \code{CloneResult}s before accumulation with a minimum clone length of 2.,
	label=lst:beforeAccumulation]
	{		
		"name": "generated by", "occurrences": [
			{originID: "/projectX/class1.java", lineNumber: "9"}
			{originID: "/projectX/class2.java", lineNumber: "9"}
			{originID: "/projectX/class3.java", lineNumber: "9"}],
		"name": "is generated by", "occurrences": [
			{originID: "/projectX/class1.java", lineNumber: "9"}
			{originID: "/projectX/class2.java", lineNumber: "9"}
			{originID: "/projectX/class3.java", lineNumber: "9"}],
		"name": "This class is generated by", "occurrences": [
			{originID: "/projectX/class1.java", lineNumber: "9"}
			{originID: "/projectX/class2.java", lineNumber: "9"}
			{originID: "/projectX/class3.java", lineNumber: "9"}],
	}
\end{lstlisting}
% !TeX root = ../main.tex
% Add the above to each chapter to make compiling the PDF easier in some editors.

\begin{lstlisting}[
	caption=The \code{CloneResult}s after accumulation with a minimum clone length of 2.,
	label=lst:afterAccumulation]
	{		
		"name": "generated by", "occurrences": [
			{originID: "/projectX/class1.java", lineNumber: "9"}
			{originID: "/projectX/class2.java", lineNumber: "9"}
			{originID: "/projectX/class3.java", lineNumber: "9"}]
	}
\end{lstlisting}

\subsection{Clustering of CloneResults}
\label{section:clustering}
During the search for generator-patterns we realized that the generated files always have the same structure, resulting from the fact that generators mostly reuse the same templates that are only filled with user-dependent information.\\
This results in the observation that most \code{CloneResult}s will have their occurrences always at the same line in the source code file, only having a small derivation coming from additional license headers or package declarations.\\
To check whether the \code{CloneResult} is likely to be a generator-pattern we iterate over all occurrences to sort out the ones that only occur once in their respective line. Furthermore we added a threshold for the maximal distance between the occurrences of ten lines, what sorts out most of the common phrases and keeps all generator-patterns.

\section{Generate links to the files}
\label{section:generateLinks}
The last step performed by the tool is to save the \code{CloneResult}s to be further processed. This gets done in three separate steps:
\begin{enumerate}
	\item Save as repository
		\subitem A \code{Java} class is generated that holds a \code{List} of \code{Strings} in which a human readable version of the presumable generator pattern is saved together with a escaped version that can be used in regex search for later usage as the generator pattern repository.
	\item Save as files
		\subitem The exact folder structure of the original projects is saved, but removing all files that do not contain the presumable generator patterns. In these newly created projects one can easily review the found occurrences and check for the algorithms reliability. Additionally, a plain \code{XML} list of all presumable generated files is saved.
	\item Save as links
		\subitem The \code{CloneResult}s are sorted by their number of occurrences and for each number a folder is created. Within these folders, we created subfolders labeled by the name of the included possible generator pattern. In these subfolders, links to all files containing the generator pattern are created.
\end{enumerate} 
\threaded{Save as files and as links}

\section{Creation of a Generator-Pattern Repository}
When the tool is finished and the raw data is saved it needs to be reviewed by hand. \\
We searched the folders including the links to the generator patterns (sorted by their number of occurrences) in descending order due to the fact that a high number of occurrences implies many repetitions of the comment, which is a possible sign of it being generated.\\
By looking at the comments including the found presumable generator patterns we found a variety of different generator patterns. Table~\ref{table:generatorPatternRepository} shows an excerpt of the actual generator-pattern repository associated to their respective generator.\\
As stated in \cite{Bernwieser2014}, the word \textit{generate} used to filter the \code{CloneResult}s is not a sufficient criterion to decide whether a source code file is generated. Due to this fact the algorithm produces many false positives and patterns that use the word \textit{generated}, but in the context that it describes that the following function or class generates something, rather than being generated itself. \\
We reviewed the presumable generator-patterns and quickly found multiple comments that mark the source code file as generated. These were collected in the Generator-Pattern Repository.\\
Based on the repository we could iteratively find more patterns. To do so we repeated all steps described in the approach, but checked the source code files directly after retrieving them from the Teamscale instance. The source code files containing one or multiple of the generator-patterns were sorted out, so the suffix-tree clone-detection has been performed on a shrinking set of source code files containing less generated code in every iteration.\\
We repeated searching for generator-patterns and adding them to the repository and the suffix-tree clone-detection until no generator-patterns were left.

% !TeX root = ../main.tex
% Add the above to each chapter to make compiling the PDF easier in some editors.
\setlength{\extrarowheight}{0.2em}
\begin{table}
	\caption{An excerpt of the Generator-Pattern Repository}
	\label{table:generatorPatternRepository}
	\begin{tabularx}{\textwidth}{c|C}
		\textbf{Generator} & \textbf{Regular Expression Pattern} \\
		\hline
		Apache Axis & This file was auto-generated from WSDL by the Apache Axis WSDL2Java emitter. \\
		Apache Cayenne & It is (probably )?a good idea to avoid changing this class manually(,)? since it (((may)|(will)) )be overwritten next time code is regenerated. If you need to make any customizations(,)? (((please use)|(put them in a)) )subclass. \\
		Apache Thrift & Autogenerated by Thrift DO NOT EDIT UNLESS YOU ARE SURE THAT YOU KNOW WHAT YOU ARE DOING \\
		CUP & The following code was generated by CUP \version \timestamp \\
		Eclipse & TODO Auto-generated method stub \\
		Java Annotation & @generated( modifiable)? \\
		JavaCC & This class is generated by JavaCC. \\
		JavaNCSS & WARNING TO \project DEVELOPERS DO NOT MODIFY THIS FILE! MODIFY THE FILES UNDER THE JAVANCSS DIRECTORY LOCATED AT THE ROOT OF THE \project PROJECT. FOLLOW THE PROCEDURE FOR MERGING THE LATEST JAVANCSS INTO \project LOCATED AT javancss/coberturaREADME.txt \\
		JAXB & This file was generated by the JavaTM Architecture for XML Binding\(JAXB\) Reference Implementation,( \version See <a href="http://java.sun.com/xml/jaxb"> http://java.sun.com/xml/jaxb</a> Any modifications to this file will be lost upon recompilation of the source schema. Generated on: \timestamp)? \\
		JFlex & NOTE: This class was automatically generated. DO NOT MODIFY. \\
		NetBeans & Do NOT modify this code. The content of this method is always regenerated by the Form Editor. \\
		SableCC & This file was generated by SableCC (\(http://www.sablecc.org/\).)? \\
		schemagen & @author Auto-generated by schemagen on \timestamp \\
		Snowball & This file was generated automatically by the Snowball to Java compiler
		
		
	\end{tabularx}
\end{table}
\setlength{\extrarowheight}{0em}










