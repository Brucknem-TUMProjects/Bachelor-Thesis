% !TeX root = ../main.tex
% Add the above to each chapter to make compiling the PDF easier in some editors.

\chapter{Terms and Definitions}\label{chapter:terms}
This chapter is to explain important terms and definitions used in the following chapters. It clears up the terminology and metrics used and gives an overview over Teamscale.
\section{Terminology}
We define the used terms and definitions to avoid the reader to be confused by arising terminology and to clarify the concrete definition we refer to when using specific terms.
\subsection{Generated code}
As proposed in \cite{Alves2011} "we consider \textit{generated code} all artifacts that are automatically created and modified by a tool using other artifacts as input. Common examples are parsers (generated from grammars), data access layers (generated from different models such as UML, database schemas or web-service specifications), mock objects or test code."\mbox{\cite[p.~11]{Alves2011}}

\subsection{Manually-maintained code}
In contrast, we consider \textit{manually-maintained code} as suggested in \cite{Alves2011} "all artifacts that have been created or modified by a developer either under development or during maintenance. This includes all artifacts created using any type of tool (e.g. editor). Configuration, experimental or temporary artifacts, if created or modified by a developer, should also be considered as \textit{manually-maintained code}."\mbox{\cite[p.~11]{Alves2011}}

\subsection{Generator pattern}
As a \textit{generator pattern} we consider all comments that a code generator adds to the generated artifacts during generation and that distinguish generated from manually-maintained code. Most code generators do add comments that are characteristic for the generator and identify the code as generated. This includes header comments preceding entire source code files as well as comments that mark single functions or datatype definitions.

\subsection{Generator pattern repository}
The \textit{generator pattern repository} is one aim of this thesis. It is a database holding the generator patterns associated to its generator. It can be used to detect generated source code files efficient and reliable.

\subsection{Clone}
We refer to \textit{clones} in the context of this thesis as text fragments that have two or more instances in comments. These instances are called clones and any two of them form a clone pair. \cite{Roy2007}\\
This includes whole comments that are identical in different source code files as well as parts of comments.

\subsection{Clone chunk}
For the use in Esko Ukkonens algorithm for the construction of suffix-trees~\cite{Ukkonen1995} the text of the comments has to be normalized in \textit{clone chunks}. The suffix-tree clone-detection approach uses a sequence of clone chunks and constructs the tree on them. To be suitable each comment is split in separate words, whereas additional information gets appended to each word that will later on be used to identify the location of the word in the original comment. This information includes the path to the original source code file, the line number in which the word originated and the programming language.

\subsection{Sentinel}
\label{section:sentinel}
A boolean comparison for equality of a sentinel and another object will always evaluate to false. It is used on the one site to separate coherent sequences of clone chunks from each other, and on the other hand to convert an implicit suffix-tree into an explicit one. A \textit{sentinel} is a subclass of the \textit{clone chunk} class. 

\subsection{Clone class}
"A \textit{clone class} is the maximal set of [comments] in which any two of the [comments] form a \textit{clone pair}."\cite[p.~3]{Bernwieser2014}\\
Table~\ref{table:cloneClass} depicts an example of the appearance of 3
clone classes.
% !TeX root = ../main.tex
% Add the above to each chapter to make compiling the PDF easier in some editors.
\setlength{\extrarowheight}{.5em}
\begin{table}
	\annotation{I dont know how to make code snippets out of it. Tried many ways but listings in tables is somewhat awful.}
	\caption{Clone Pair and Clone Class}
	\label{table:cloneClass}

%	\begin{tabular}{l|l}
%		\begin{lstlisting}[frame=none, numbers=none]
%			public void test(){ ...	}
%		\end{lstlisting} 
%		
%		&
%		
%		\begin{lstlisting}[frame=none, numbers=none]
%		public void test(){ ...	}
%		\end{lstlisting}
%	\end{tabular}		
	\begin{tabularx}{1.0\textwidth}{C|C|C}
%		\hline
		\textbf{Fragment 1} & \textbf{Fragment 2} & \textbf{Fragment 3} \\
		\hline
		
		\cellcolor{TUMAccentLightBlue}/* The following code was generated by \dots */ &
		\cellcolor{TUMAccentLightBlue}/* The following code was generated by \dots */ &
		/** This character denotes the end of file */ \\
		\hline
				
		\cellcolor{TUMAccentBlue}/* The content of this method is always regenerated */ &
		\cellcolor{TUMAccentBlue}/* The content of this method is always regenerated */ &
		\cellcolor{TUMAccentBlue}/* The content of this method is always regenerated */ \\
		\hline
		
		/** Translates characters to character classes */ &
		\cellcolor{TUMAccentLightBlue} /** This class is a scanner generated by \dots */ &
		\cellcolor{TUMAccentLightBlue} /** This class is a scanner generated by \dots */ \\
	\end{tabularx}
\end{table}

\section{Metrics}
The \textit{Lines of Code (LOC)} metric represents the size of a source code file so that it can be compared to other files easily. We use it to calculate the overall sizes of the different code categories to compare the proportions they yield in total. Additionally, we use it to calculate the average file size of generated and manually maintained source code.

\section{Teamscale}
\label{section:teamscale}
\textit{Teamscale} is developed by the CQSE GmbH which was founded in 2009 as spin-off of Technical University Munich (TUM). They offer innovative consulting and products to help their customers evaluate, control and improve their software quality.\\
Their main product is Teamscale which is a tool to monitor the quality of code with a variety of static code analyses. It helps to enhance the quality of code over time and is personally configurable to allow users to focus on personal quality goals and keep an eye on the current quality trend.\\
The supported programming languages of Teamscale are \textit{ABAP, Groovy, Matlab, Simulink/StateFlow, Ada, Gosu, Open CL, SQLScript, C\#, IEC 61131-3 ST, OScript, Swift, C/C++, Java, PHP, TypeScript, Cobol, JavaScript, PL/SQL, Visual Basic .NET, Delphi, Kotlin, Python, Xtend, Fortran, Magik} and \textit{Rust}.\\
One major aspect in the quality analysis performed by Teamscale is the \textit{clone-detection}. With it duplicated code created by copy \& paste can be found automatically.

%\subsection{Suffix-tree clone-detection}
%In \cite{Ukkonen1995}, "an on-line algorithm is presented for constructing the suffix-tree for a given string in time linear in the length of the string."\cite[p.~1]{Ukkonen1995}\\
%The algorithm has been extended to be usable in this thesis to detect clones among comments in source code.

\subsection{Lexer}
A tokenizer, also called lexical scanner (short: \textit{Lexer}), is a program that splits plain text in a sequence of logical concatenated units, so called tokens. The plain text tokenized by the lexer can be anything, but we will restrict it to source code. Teamscale includes a variety of lexers for every supported programming language. 

\subsection{Token}
\label{section:token}
Tokens in the context of Teamscale are objects that are returned as a sequence by the lexer. They provide a data structure holding the main properties of the smallest possible units a source code file can be split into. An overview of these properties is shown in table~\ref{table:tokenProperties}.
% !TeX root = ../main.tex
% Add the above to each chapter to make compiling the PDF easier in some editors.
\setlength{\extrarowheight}{.5em}
\begin{table}
	\caption{The properties that define a token.}
	\label{table:tokenProperties}
	\begin{tabularx}{\textwidth}{c|L}
		\textbf{Property} & \textbf{Description} \\
		\hline
		Text & The original input text of the token, copied verbatim from the source. \\
		Offset & The number of characters before this token in the text. \\
		Line number & The line in which the token is found in the original file. \\
%		End Offset & The number of characters before the end of this token in the text (i.e. the 0-based index of the last character, inclusive). \\
		Origin ID & The string that identifies the origin of this token. This can, e.g., be a uniform path to the resource. Its actual content depends on how the token gets constructed. \\
		Type & The type of the token. \\
%		Language & The programming language in which the source code file is written that contains the token.		
	\end{tabularx}
\end{table}
\setlength{\extrarowheight}{0em}

\subsection{Token class}
A token is always an element of one of the token classes \textit{LITERAL, KEYWORD, IDENTIFIER, DELIMITER, COMMENT, SPECIAL, ERROR, WHITESPACE} or \textit{SYNTHETIC}. These are mutually exclusive sets that wrap all possible types a token can adopt.

\subsection{Token type}
A token always adopts one single type. These range from the simple \textit{INTEGER\_LITERAL} over \textit{HEADER} to \textit{DOCUMENTATION\_COMMENT}.
