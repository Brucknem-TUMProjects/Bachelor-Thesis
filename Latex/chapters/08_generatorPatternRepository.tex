% !TeX root = ../main.tex
% Add the above to each chapter to make compiling the PDF easier in some editors.

\chapter{Generator-Pattern Repository}\label{chapter:generatorPatternRepository}
The following pages display the current status of the generator-pattern repository and its usage. It is optimized for the use with the regular expression framework used by the Java programming language.\\
When we applied the approach presented in this thesis, we normalized the comments into \code{CloneChunk}s. We did so by using the procedure presented in section~\ref{section:convertToCloneChunks}.\\
The resulting \code{List} of \code{CloneChunk}s represents a \code{List} of single words, whereas all white spaces and words without value have been removed. The resulting generator-pattern repository thus holds the generator-patterns without any formatting and programming language specific comment markers.

\section{Usage of the generator-pattern repository}
The easiest way to use the repository is by compiling the generator-patterns into regular expressions and normalizing the source code files when searching for the generator-patterns. This can be done in two steps:

\begin{enumerate}
	\item To use the generator-pattern \code{String}s one has to at first replace the placeholders \textit{<Version>, <Timestamp>, <Project>, <Class>} and \textit{<Link>}. The current observations have shown that a good replacement is the regex \textit{(.*)} that represents a sequence of any character with a variable length. Afterwards the resulting \code{String}s can be compiled into \code{Pattern}s and used in the regex framework.
	\item Due to the fact that the formatting of the generator-patterns has been removed to have them as universally as possible the source code files have to be normalized. \\
	At first one has to split the source code files at every line~break and white~space. Afterwards, every word is checked if it is valuable (see section~\ref{section:convertToCloneChunks}). The remaining words get merged into one \code{String}, whereas the words are delimited by a single white~space. An exemplary Java implementation is displayed in listing~\ref{lst:normalizeFunction}.\\
	On the resulting  \code{String} a regex matching can be performed with the generator-patterns.
\end{enumerate}
% !TeX root = ../main.tex
% Add the above to each chapter to make compiling the PDF easier in some editors.

\begin{lstlisting}[
	caption=Java function to normalize a source code file to be easily usable with the generator-pattern repository.,
	label=lst:normalizeFunction]
static String normalize(String toNormalize) {
	StringBuilder sb = new StringBuilder();

	for (String line : toNormalize.split("\\R")) {
		for (String word : line.split("\\s+")) {
			if (isValuable(word)) {
				sb.append(word + " ");
			}
		}
	}

	return sb.toString();
}	

static boolean isValuable(String word) {
	for (Character letter : word.toCharArray()) {
		if (Character.isAlphabetic(letter) || Character.isDigit(letter)) {
			return true;
		}
	}
	
	return false;
}
	
\end{lstlisting}

\cleardoublepage{}
\section{Qualitas Corpus}
The following tables picture the set of generator-patterns that was found using the algorithm on the Qualitas Corpus.
% !TeX root = ../main.tex
% Add the above to each chapter to make compiling the PDF easier in some editors.
\setlength{\extrarowheight}{0.2em}
\begin{table}
	\caption{The generator-patterns found in the Qualitas Corpus.}
	\label{table:generatorPatternRepositoryQC}
	\begin{tabularx}{\textwidth}{c|C}
		\textbf{Generator} & \textbf{Regular Expression Pattern} \\
		\hline
		Ant generate & perform the desired modifications there and re-generate by "ant generate". \\
		ANTLR & ANTLR \version \\
		Apache Axis & This file was auto-generated from WSDL by the Apache Axis (\version \timestamp )?WSDL2Java emitter. \\
		Apache Cayenne & It is (probably )?a good idea to avoid changing this class manually(,)? since it (((may)|(will)) )be overwritten next time code is regenerated. If you need to make any customizations(,)? (((please use)|(put them in a)) )subclass. \\
		Apache Thrift & Autogenerated by Thrift DO NOT EDIT UNLESS YOU ARE SURE THAT YOU KNOW WHAT YOU ARE DOING \\
		Castor & This class was automatically generated with <a href="http://www.castor.org">Castor \version</a>, using an XML Schema. \\
		com.ibm.icu.text.Normalizer & THIS FILE WAS MACHINE GENERATED DO NOT EDIT BY HAND \\
		Compiere &  Generated (Model|VO) ((for)|(- DO NOT CHANGE)) \\
		CUP & CUP \version generated parser. @version \timestamp \\
		CUP & CUP generated class containing symbol constants. \\
		CUP & The following code was generated by CUP \\
		Hibernate & Auto-generated using Hibernate hbm2java tool. \\
		IDL-to-Java compiler & Generated by the IDL-to-Java compiler \\
		Java Annotation & @generated \\
		JavaCC & Generated By:JavaCC: Do not edit this line. \\
		JavaCC & The actual parser implementing this interface is generated from JavaCC grammar. \\
		JavaCC & This class is generated by JavaCC. \\
		JavaCC JJTree & Generated By:JJTree&JavaCC: Do not edit this line. \\
		JavaNCSS & WARNING TO \project DEVELOPERS DO NOT MODIFY THIS FILE! MODIFY THE FILES UNDER THE JAVANCSS DIRECTORY LOCATED AT THE ROOT OF THE \project PROJECT. FOLLOW THE PROCEDURE FOR MERGING THE LATEST JAVANCSS INTO \project LOCATED AT javancss/coberturaREADME.txt \\
		JAXB & This file was generated by the JavaTM Architecture for XML Binding\(JAXB\) Reference Implementation,( \version See <a href="http://java.sun.com/xml/jaxb"> http://java.sun.com/xml/jaxb</a> Any modifications to this file will be lost upon recompilation of the source schema. Generated on: \timestamp)? \\
		JAXRPC & This class was generated by the JAXRPC SI, do not edit. \\
		JAXWS & This class was generated by the JAXWS SI. \\
		Jena vocabulary generator & Vocabulary Class generated by Jena vocabulary generator \\
		JFlex & NOTE: This class was automatically generated. DO NOT MODIFY. \\
		JFlex & The following code was generated by JFlex \version on \timestamp \\
		JFlex & This class is a scanner generated by <a href="http://www.jflex.de/">JFlex</a> \version on \timestamp from the specification file  \\
		JJTree & <p>Generated by JJTree</p> \\
		JJTree & Automatically generated by JJTree \\
		JJTree & Generated By:JJTree: Do not edit this line. \\
		Junit & suite method automatically generated by JUnit module \\
		Junit & TODO review the generated test code and remove the default call to fail. \\
		Moman FineNight & The following code was generated with the moman/finenight pkg \\
		NetBeans Eclipse & Do NOT modify this code. The content of this method is always regenerated by the Form( )?Editor. \\
		org.eclipse.emf.examples.generator.validator & It was generated by the org.eclipse.emf.examples.generator.validator plug-in to illustrate how EMF's code generator can be extended. \\
		org.geotools.resources.IndexedResourceCompiler & THIS IS AN AUTOMATICALLY GENERATED FILE. \\
		SableCC & This file was generated by SableCC (\(http://www.sablecc.org/\).)? \\
		schemagen & @author Auto-generated by schemagen on \timestamp \\
		Snowball & This class was automatically generated by a Snowball to Java compiler It implements the stemming algorithm defined by a snowball script. \\
		Snowball & This file was generated automatically by the Snowball to Java compiler \\
		Together & Generated by Together \\
		Unknown & @author Jorg Janke \(generated\) \\
		Unknown & Generated file - Do not edit! \\
		Unknown & NOTE: This source is automatically generated please do not modify this file.  Either subclass or remove the record \\
		Unknown & this code is autogenerated - you shouldnt be modifying it! \\
		Unknown & This file has been automatically generated, DO NOT EDIT \\
		Unknown & WARNING: This file was automatically generated. Do not edit it directly, \\
		Xdoclet & Generated by Xdoclet - Do not edit! \\		
	\end{tabularx}
\end{table}
\setlength{\extrarowheight}{0em}
% !TeX root = ../main.tex
% Add the above to each chapter to make compiling the PDF easier in some editors.
\setlength{\extrarowheight}{0.2em}
\begin{table}
	\label{table:generatorPatternRepository_QC_01}
	\begin{tabularx}{\textwidth}{c|C}
		\textbf{Generator} & \textbf{Regular Expression Pattern} \\
		\hline			
		JavaNCSS & WARNING TO \project DEVELOPERS DO NOT MODIFY THIS FILE! MODIFY THE FILES UNDER THE JAVANCSS DIRECTORY LOCATED AT THE ROOT OF THE \project PROJECT. FOLLOW THE PROCEDURE FOR MERGING THE LATEST JAVANCSS INTO \project LOCATED AT javancss/coberturaREADME.txt \\
		JAXB & This file was generated by the JavaTM Architecture for XML Binding\textbackslash(JAXB\textbackslash) Reference Implementation,( \version See <a href="http://java.sun.com/xml/jaxb"> http://java.sun.com/xml/jaxb</a> Any modifications to this file will be lost upon recompilation of the source schema. Generated on: \timestamp)? \\
		JAXRPC & This class was generated by the JAXRPC SI, do not edit. \\
		JAXWS & This class was generated by the JAXWS SI. \\
		Jena vocabulary generator & Vocabulary Class generated by Jena vocabulary generator \\
		JFlex & NOTE: This class was automatically generated. DO NOT MODIFY. \\
		JFlex & The following code was generated by JFlex \version on \timestamp  \\
		JFlex & This class is a scanner generated by <a href="http://www.jflex.de/">JFlex</a> \version on \timestamp from the specification file  \\
		JJTree & <p>Generated by JJTree</p> \\
		JJTree & Automatically generated by JJTree \\
		JJTree & Generated By:JJTree: Do not edit this line. \\
		Junit & suite method automatically generated by JUnit module \\
		Junit & TODO review the generated test code and remove the default call to fail. \\
		Moman FineNight & The following code was generated with the moman/finenight pkg \\
		NetBeans Eclipse & Do NOT modify this code. The content of this method is always regenerated by the Form( )?Editor. \\
	\end{tabularx}
\end{table}
\setlength{\extrarowheight}{0em}
% !TeX root = ../main.tex
% Add the above to each chapter to make compiling the PDF easier in some editors.
\setlength{\extrarowheight}{0.2em}
\begin{table}
	\caption{The generator-patterns found in the Qualitas Corpus.}
	\label{table:generatorPatternRepository_QC_02}
	\begin{tabularx}{\textwidth}{c|C}
		\textbf{Generator} & \textbf{Regular Expression Pattern} \\
		\hline
		\makecell{org.eclipse.\\emf.examples.\\generator.validator} & It was generated by the org.eclipse.emf.examples.generator.validator plug-in to illustrate how EMF's code generator can be extended. \\
		\makecell{org.geotools.\\resources.\\IndexedResourceCompiler} & THIS IS AN AUTOMATICALLY GENERATED FILE. \\
		SableCC & This file was generated by SableCC ((http://www.sablecc.org/).)? \\
		schemagen & @author Auto-generated by schemagen on \timestamp \\
		Snowball & This class was automatically generated by a Snowball to Java compiler It implements the stemming algorithm defined by a snowball script. \\
		Snowball & This file was generated automatically by the Snowball to Java compiler \\
		Together & Generated by Together \\
		Unknown & @author Jorg Janke \textbackslash(generated\textbackslash) \\
		Unknown & Generated file - Do not edit! \\
		Unknown & NOTE: This source is automatically generated please do not modify this file.  Either subclass or remove the record \\
		Unknown & this code is autogenerated - you shouldnt be modifying it! \\
		Unknown & This file has been automatically generated, DO NOT EDIT \\
		Unknown & WARNING: This file was automatically generated. Do not edit it directly, \\
		Xdoclet & Generated by Xdoclet - Do not edit! \\
	\end{tabularx}
\end{table}
\setlength{\extrarowheight}{0em}

\cleardoublepage{}
\section{Random projects}
The following tables picture the set of generator-patterns that was found using the algorithm on the random projects.
% !TeX root = ../main.tex
% Add the above to each chapter to make compiling the PDF easier in some editors.
\setlength{\extrarowheight}{0.2em}
\begin{table}
	\label{table:generatorPatternRepository_Git}
	\begin{tabularx}{\textwidth}{c|C}
		\textbf{Generator} & \textbf{Regular Expression Pattern} \\
		\hline
		ANTLR & or the code generated by ((PCCTS)|(SORCERER)) \\
		Apache Thrift & Autogenerated by Thrift Compiler \version \\
		cake build & This file is generated by `cake build`, do not edit it by hand. \\
		Celerio & Source code generated by Celerio, a Jaxio product \\
		Coco/R & This file was generated with Coco/R Java, version: \version \\
		Doctrine ORM & This class was generated by the Doctrine ORM. \\
		flex & A lexical scanner generated by flex \\
		generateDS.py & Generated \timestamp by generateDS.py version \version \\
		get\_mozilla\_ciphers.py. & This file was automatically generated by get\_mozilla\_ciphers.py. \\
		GNATtest & Do not edit any part of it, see GNATtest documentation for more details. \\
		GNATtest & This package has been generated automatically by GNATtest. \\
		Grunt & This file is autogenerated via the `commonjs` Grunt task. You can require() this file in a CommonJS environment. \\
		JastAdd2 & This file was generated with JastAdd2 \\
		JAX-WS & This class was generated by the JAX-WS RI. \\
		JAXB-RI & This file was generated by JAXB-RI \version \\
		Jflex & The following code was generated by JFlex \version \\
		jni4Android & This file is automatically generated by jni4android, do not modify. \\
		MyEclipse Persistence Tools & Mapping file autogenerated by MyEclipse Persistence Tools \\
		PrestaShop & Do not edit or add to this file if you wish to upgrade PrestaShop to newer versions in the future. \\
	\end{tabularx}
\end{table}
\setlength{\extrarowheight}{0em}
% !TeX root = ../main.tex
% Add the above to each chapter to make compiling the PDF easier in some editors.
\setlength{\extrarowheight}{0.2em}
\begin{table}
	\label{table:generatorPatternRepository_Git_01}
	\begin{tabularx}{\textwidth}{c|C}
		\textbf{Generator} & \textbf{Regular Expression Pattern} \\
		\hline
		Pygments & This file is generated by itself. \\
		RelaxNGCC & this file is generated by RelaxNGCC \\
		SableCC & This file was generated by SableCC's ObjectMacro. \\
		scripts/unicode.py & NOTE: The following code was generated by "scripts/unicode.py", do not edit directly \\
		SWIG & Do not make changes to this file. Modify the SWIG interface file instead. \\
		SWIG & This file is generated by SWIG. Please do *not* modify by hand. \\
		SWIG & This file was automatically generated by SWIG version \version \\
		TestGenerator & This file ((is)|(and method are)) generated by TestGenerator, any edits will be overwritten by the next generation. \\
		TI PinMux & This file was automatically generated on \timestamp by TI PinMux version \version \\
		ucd-generate & DO NOT EDIT THIS FILE. IT WAS AUTOMATICALLY GENERATED BY: \\
		Unknown & <p><b> Auto-generated, do not edit. </b></p> \\
		Unknown & DO NOT EDIT THIS FILE - it is machine generated \\
		Unknown & Do not edit this file directly, it is built from the sources at \link \\
		Unknown & This code was generated by \\
		Unknown & This file has been automatically generated from \class \\	
	\end{tabularx}
\end{table}
\setlength{\extrarowheight}{0em}