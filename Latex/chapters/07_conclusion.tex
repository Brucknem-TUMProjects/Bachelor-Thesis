% !TeX root = ../main.tex
% Add the above to each chapter to make compiling the PDF easier in some editors.

\chapter{Conclusion}\label{chapter:conclusion}
We presented an algorithm that uses a suffix-tree clone-detection approach to find clones among the comments in source code. By using this approach we implemented a semi-automatic way to build up a generator-pattern repository.\\
We used a Teamscale server that extracts the comments out of the source code files that have been uploaded. These comments are normalized to be usable in a modified suffix-tree algorithm. Based on the suffix-tree we searched for clones and filtered the results to make them processable manually. The filtered results have been saved and reviewed by hand to build up the generator-pattern repository. This repository holds a multitude of generator-patterns associated to their respective generator.\\
We used the algorithm on a reference data set (Qualitas Corpus) and a huge, randomly composed collection of open source projects we gathered from \href{https://github.com}{GitHub} to find a wide range of generator-patterns for a multitude of generators and many different programming languages. Resulting we filled the repository with a total of 82 generator-patterns.\\
We used the found patterns on the data sets to categorize the source code in \textit{generated} and \textit{manually maintained} code. Hereby we found very different proportions of generated code ranging from 0\% for some whole programming languages up to 75\% for single projects. This resulted in a average of 5\% to 20\% for generated code in projects that use code generators.\\
In conclusion we saw that the suffix-tree clone-detection approach to find generator-patterns is promising in the search for generated code and that the generator-pattern repository is an efficient and reliable attempt for automatic code classification.