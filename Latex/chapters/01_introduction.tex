% !TeX root = ../main.tex
% Add the above to each chapter to make compiling the PDF easier in some editors.

\chapter{Introduction}\label{chapter:introduction}

Source code of software can be categorized regarding to its role in the maintainability of a software system. The biggest categories besides manually produced code, which is written and maintained by hand of a developer, are generated code and test-code. In many systems up to 50\% of the source code arise in these categories.

Static analysis detects problems in the quality of source code. At the same time the code category is essential for the relevance of the examined quality criterion. Security flaws and performance problems which are crucial in production code can be irrelevant in generated code since it is not directly edited during maintenance. To enhance the relevance and significance of the results of static analysis the category of the examined source code has to be taken into account.

This applies particularly in benchmarks that investigate the frequency of the occurrence of quality defects and the distribution of metric values in a variety of software projects. Since a manual classification of source code in a multitude of projects is not feasible in practice an automated approach is necessary.

This work targets the conception and prototypical implementation of an automatic detection of generated code which includes following steps:

\begin{itemize}
	\item Prototypical implementation of heuristics to detect generated code which use techniques of clone detection on the comments that are extracted from the source code.
	\item A list of heuristics that detect generated code of several code generators will be derived by means of the comments. Therefore a generator pattern repository will be created.
	\item The completeness and accuracy of the developed heuristics will be evaluated on a reference data set.
	\item Automatic classification of source code in a variety of open source systems and evaluation of the amount of generated code. 
\end{itemize}